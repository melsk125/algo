\documentclass[book]{jlreq}

\usepackage{amsthm}

\newtheorem{problem}{問題}[chapter]
\newenvironment{solution}
  {\begin{proof}[回答:]}
  {\end{proof}} 

\title{『アルゴリズムイントロダクション 第4版』章末問題}
\author{panot}

\begin{document}

\maketitle

\chapter{計算におけるアルゴリズムの役割}

\begin{problem}
  各関数$f(n)$と時間$t$に対して、アルゴリズムが問題を解くのに$f(n)$マイクロ秒かかる
  とき、$t$時間で解くことができる最大の問題サイズ$n$を求めよ。
\end{problem}
\begin{solution}
  このような計算結果になる。
  \begin{table}
    \begin{tabular}{|c|c|c|c|c|c|c|c|}
      \hline
      & 1秒 & 1分 & 1時間 & 1日 & 1ヶ月 & 1年 & 1世紀 \\ \hline
      $\lg n$ & $10^{301}$ & $10^{18061}$ & $10^{1083707}$ & $10^{26008991}$ &
        $10^{78 0269748}$ & $10^{9363236985}$ & $10^{936323698513}$ \\ \hline
      $\sqrt{n}$ & $10^6$ & $3.6\times 10^9$ & $1.30 \times 10^{13}$ &
        $7.46 \times 10^{15}$ & $6.72 \times 10^{18}$ & $9.67 \times 10^{20}$ &
        $9.67 \times 10^{22}$ \\ \hline
      $n$ & $10^3$ & $6 \times 10^4$ & $3.6 \times 10^6$ & $8.64 \times 10^7$ &
        $2.59 \times 10^9$ & $3.11 \times 10^{10}$ & $3.11 \times 10^{12}$
        \\ \hline
      $n\lg n$ & $140$ & $4895$ & $204094$ & $3.94 \times 10^6$ &
        $9.77 \times 10^7$ & $1.04 \times 10^9$ & $8.56 \times 10^{10}$
        \\ \hline
      $n^2$ & $31$ & $244$ & $1897$ & $9295$ & $50911$ & $176363$ &
        $1.76 \times 10^6$ \\ \hline
      $n^3$ & $10$ & $39$ & $153$ & $442$ & $1373$ & $3144$ & $14597$ \\ \hline
      $2^n$ & $9$ & $15$ & $21$ & $26$ & $31$ & $34$ & $41$ \\ \hline
      $n!$ & $6$ & $8$ & $9$ & $11$ & $12$ & $13$ & $15$ \\
      \hline
    \end{tabular}
    \caption{計算結果}
  \end{table}
\end{solution}

\end{document}
