% LuaLaTeX文書; 文字コードはUTF-8
\documentclass[unicode,12pt,aspectratio=169]{beamer} % 'unicode'が必要
\usepackage{luatexja} % 日本語したい
% \usepackage[ipaex]{luatexja-preset} % IPAexフォントしたい
% \renewcommand{\kanjifamilydefault}{\gtdefault} % 既定をゴシック体に


\usepackage{geometry}
\usepackage{pxrubrica}
\usepackage{import}

% あとは欧文の場合と同じ
\usetheme{boxes}

\title{精読『アルゴリズムイントロダクション 第4版』}
\subtitle{第1回}
\author{Panot}
\date{最終更新:\today}

\begin{document}

\begin{frame}
 \titlepage{}
\end{frame}

\begin{frame}{今日の範囲}
  第1章:計算におけるアルゴリズムの役割
  第2章:さあ、始めよう
  第3章:実行時間の特徴づけ
\end{frame}

\begin{frame}{今日の内容}
  \begin{itemize}
    \item アルゴリズムとは?
    \item なぜアルゴリズムの研究が必要なのか?
    \item アルゴリズム解析
      \begin{itemize}
        \item 挿入ソート
        \item マージソート
      \end{itemize}
    \item 「オーダーノーテーション」
      \begin{itemize}
        \item 直感
        \item 定義
      \end{itemize}
  \end{itemize}
\end{frame}

\begin{frame}{アルゴリズムとは}
  \begin{itemize}
    \item 明確に定義された手続き
    \item 値または値の集合を\textbf{入力}として取る
    \item ある値または値の集合を\textbf{出力}して生成する
  \end{itemize}
\end{frame}

\begin{frame}{アルゴリズムとは}
  \begin{itemize}
    \item \textbf{計算問題}を解く道具として
    \item \textbf{正当な}アルゴリズム
    \item すべての\textbf{問題のインスタンス}に対して
    \begin{itemize}
      \item 常に\textbf{停止}する
      \item 出力が\textbf{正しい}
    \end{itemize}
  \end{itemize}
\end{frame}

\begin{frame}{ソーティング問題}
  \textbf{入力}:$n$個の数の列$\left\langle a_1, a_2, \ldots, a_n\right\rangle$

  \textbf{出力}:$a_1'\leq a_2'\leq \ldots\leq a_n'$を満たす入力列の置換(並べ換え)
  $\left\langle a_1', a_2', \ldots, a_n'\right\rangle$

  \textbf{問題のインスタンス}の例:
  $\left\langle 31, 41, 59, 26, 41, 58\right\rangle$

  \textbf{正しい出力}:$\left\langle 26, 31, 41, 41, 58, 59\right\rangle$
\end{frame}

\begin{frame}{実行時間の比較}
  各関数$f(n)$と時間$t$に対して、アルゴリズムが問題を解くのに$f(n)$マイクロ秒かかる
  とき、$t$時間で解くことができる最大の問題サイズ$n$を求めよ。

  \begin{center}
    \begin{tabular}{|c|c|c|c|c|c|c|c|}
      \hline
      & 1秒 & 1分 & 1時間 & 1日 & 1ヶ月 & 1年 & 1世紀 \\ \hline
      $\lg n$ & & & & & & & \\ \hline
      $\sqrt{n}$ & & & & & & & \\ \hline
      $n$ & & & & & & & \\ \hline
      $n\lg n$ & & & & & & & \\ \hline
      $n^2$ & & & & & & & \\ \hline
      $n^3$ & & & & & & & \\ \hline
      $2^n$ & & & & & & & \\ \hline
      $n!$ & & & & & & & \\
      \hline
    \end{tabular}
  \end{center}
\end{frame}

\begin{frame}{実行時間の比較}
  各関数$f(n)$と時間$t$に対して、アルゴリズムが問題を解くのに$f(n)$マイクロ秒かかる
  とき、$t$時間で解くことができる最大の問題サイズ$n$を求めよ。

  {\scriptsize \begin{center}
    \begin{tabular}{|c|c|c|c|c|c|c|c|}
      \hline
      & 1秒 & 1分 & 1時間 & 1日 & 1ヶ月 & 1年 & 1世紀 \\ \hline
      $\lg n$ & $10^{301}$ & $10^{18061}$ & $10^{1083707}$ & $10^{26008991}$ &
        $10^{78 0269748}$ & $10^{9363236985}$ & $10^{936323698513}$ \\ \hline
      $\sqrt{n}$ & $10^6$ & $3.6\times 10^9$ & $1.30 \times 10^{13}$ &
        $7.46 \times 10^{15}$ & $6.72 \times 10^{18}$ & $9.67 \times 10^{20}$ &
        $9.67 \times 10^{22}$ \\ \hline
      $n$ & $10^3$ & $6 \times 10^4$ & $3.6 \times 10^6$ & $8.64 \times 10^7$ &
        $2.59 \times 10^9$ & $3.11 \times 10^{10}$ & $3.11 \times 10^{12}$
        \\ \hline
      $n\lg n$ & $140$ & $4895$ & $204094$ & $3.94 \times 10^6$ &
        $9.77 \times 10^7$ & $1.04 \times 10^9$ & $8.56 \times 10^{10}$
        \\ \hline
      $n^2$ & $31$ & $244$ & $1897$ & $9295$ & $50911$ & $176363$ &
        $1.76 \times 10^6$ \\ \hline
      $n^3$ & $10$ & $39$ & $153$ & $442$ & $1373$ & $3144$ & $14597$ \\ \hline
      $2^n$ & $9$ & $15$ & $21$ & $26$ & $31$ & $34$ & $41$ \\ \hline
      $n!$ & $6$ & $8$ & $9$ & $11$ & $12$ & $13$ & $15$ \\
      \hline
    \end{tabular}
  \end{center}}
\end{frame}

\end{document}
